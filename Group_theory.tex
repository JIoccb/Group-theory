\documentclass[a4paper, 14pt]{extarticle}
\usepackage{titlesec}
\usepackage{tocloft}
\usepackage[nottoc]{tocbibind}


\renewcommand{\cftsecfont}{\normalsize\bfseries}
\renewcommand{\cftsecpagefont}{\normalsize\bfseries} 
\titleformat{\section}{\large\bfseries}{\thesection}{1em}{}
\renewcommand{\cftsecleader}{\cftdotfill{\cftdotsep}}

\usepackage[left=3cm, right=1.5cm, top=2cm, bottom=2cm]{geometry}
\usepackage{titleps} % Колонтитулы
\usepackage{indentfirst} % Красная строка после заголовка
\usepackage{bm}
\usepackage[english, russian]{babel}
\usepackage{stmaryrd} % Стрелки в формулах
\usepackage{upgreek, tipa} % Красивые греческие буквы
\usepackage{amsmath, amsfonts, amssymb, amsthm, mathtools} % ams пакеты для математики, табуляции
\usepackage{nicematrix} % Особые матрицы pNiceArray
\usepackage{ gensymb }
\usepackage{setspace}
\usepackage{dsfont}
\usepackage{mathrsfs}
\usepackage{calrsfs}
\usepackage{amsmath}
\usepackage{amsfonts}
\usepackage{amssymb}
\usepackage{amsthm}
\usepackage[affil-it]{authblk}
\usepackage{ulem}
\usepackage{tikz}
\usepackage{verbatim}
\usetikzlibrary{arrows.meta, positioning}


\linespread{1.25} % Межстрочный интервал
\setlength{\parindent}{1.25cm} % Табуляция
\setlength{\parskip}{0cm}

% Добавляем гипертекстовое оглавление в PDF
\usepackage[
bookmarks=true, colorlinks=true, unicode=true,
urlcolor=blue, linkcolor=black, anchorcolor=black,
citecolor=black, menucolor=black, filecolor=black,
]{hyperref}

% Убрать переносы слов
\tolerance=1
\emergencystretch=\maxdimen
\hyphenpenalty=10000
\hbadness=10000

\newpagestyle{main}{
	% Верхний колонтитул
	\setheadrule{0cm} % Размер линии отделяющей колонтитул от страницы
	\sethead{}{}{} % Содержание {слева}{по центру}{справа}
	% Нижний колонтитул
	\setfootrule{0cm} % Размер линии отделяющей колонтитул от страницы
	\setfoot{}{}{\thepage} % Содержание {слева}{по центру}{справа}
}

% НОВЫЕ КОМАНДЫ
\newcommand{\deriv}[2]{\frac{\partial #1}{\partial #2}}
\newcommand{\n}{\par}
\newcommand{\percent}{\mathbin{\%}}
\newcommand{\integers}{\mathbb{Z}}
\newcommand{\naturals}{\mathbb{N}}
\newcommand{\rationals}{\mathbb{Q}}
\newcommand{\real}{\mathbb{R}}
\newcommand{\complex}{\mathbb{C}}
\newcommand{\algebraic}{\mathbb{A}}

\newcommand{\GL}{\mathrm{GL}}
\newcommand{\SL}{\mathrm{SL}}
\newcommand{\Orth}{\mathrm{O}}
\newcommand{\SOrth}{\mathrm{SO}}
\newcommand{\Unit}{\mathrm{U}}
\newcommand{\SUnit}{\mathrm{SU}}
\newcommand{\ord}{\mathrm{ord}}
\newcommand{\CS}{\mathrm{CS}}
\newcommand{\suchthat}{{:}{ } \ }
\newcommand{\im}{\mathrm{Im} \,}
\newcommand{\id}{\mathrm{id}}
\newcommand{\Isom}{\mathrm{Isom}}
\newcommand{\Rot}{\mathrm{R}}
\newcommand{\Sym}{\mathrm{S}}


\DeclareRobustCommand{\divby}{%
	\mathrel{\text{\vbox{\baselineskip.65ex\lineskiplimit0pt\hbox{.}\hbox{.}\hbox{.}}}}%
}

\DeclareRobustCommand{\ndivby}{%
	\mathrel{\text{%
			\vbox{\baselineskip.65ex\lineskiplimit0pt\hbox{.}\hbox{.}\hbox{.}}%
			\kern-1.5ex\hbox{/}%
	}}%
}

% ПЕРЕГРУЗКА УЖЕ СУЩЕСТВУЮЩИХ КОМАНД
\renewcommand{\epsilon}{\varepsilon} % Заменить знак эпсилон
\renewcommand{\phi}{\varphi}
\renewcommand{\kappa}{\varkappa}
\renewcommand{\lambda}{\uplambda}
\renewcommand{\mod}{\mathrm{mod} \,}
\renewcommand\qedsymbol{$\blacksquare$}

\theoremstyle{definition}

\newtheorem*{exmpl}{\textit{Пример}}
\newtheorem*{exmpl*}{\textit{Пример}}
\newtheorem*{exmpls}{\textit{Примеры}}
\newtheorem*{remark}{\textit{Замечание}}
\newtheorem{definition}{Определение}
\newtheorem*{definition*}{Определение}


\theoremstyle{plain}

\newtheorem{theorem}{Теоерма}
\newtheorem*{theorem*}{Теорема}
\numberwithin{theorem}{section}
\numberwithin{definition}{section}
\newtheorem{statement}{Утверждение}
\newtheorem*{statement*}{Утверждение}
\numberwithin{statement}{section}
\newtheorem{lemma}{Лемма}
\newtheorem*{lemma*}{Лемма}
\numberwithin{lemma}{section}
\newtheorem{consequence}{Следствие}
\newtheorem*{consequence*}{Следствие}
\numberwithin{consequence}{section}
\title{Введение в теорию групп}
\author{Артём Рашевский}
\affil{ФКН ВГУ}
\date{2025}
\begin{document}
	
	\maketitle
	\thispagestyle{empty}
	\newpage
	\tableofcontents
	\newpage
	\section{Бинарные операции. Полугруппы, моноиды и группы}
	\begin{definition}
		Пусть $M$~--- непустое множество. \textit{Бинарной операцией} $\circ$ на множестве $M$ называется отображение 
		${\circ{:} \ M \times M \rightarrow M}$, ${\ \forall a,b \in M{:} \ (a, b) \mapsto a \circ b}$.
	\end{definition}
	
	Множество с бинарной операцией обычно обозначают $(M, \circ).$
	
	\begin{definition}
		Множество с бинарной операцией $(M, \circ)$ называется \textit{полугруппой}, если данная бинарная операция
		ассоциативна, т.е.
		\begin{equation*}
			\forall a, b, c \in M{:} \ a \circ (b \circ c) = (a \circ b) \circ c.
		\end{equation*}
	\end{definition}
	
	\begin{definition}
		Полугруппа $(M, \circ)$ называется \textit{моноидом}, если в ней есть \textit{нейтральный элемент}, т.е.
		\begin{equation*}
			\exists e \in M \suchthat \forall a \in M{:} \ e \circ a = a \circ e = a.
		\end{equation*}
	\end{definition}
	
	\begin{definition}
		Моноид $(M, \circ)$ называется \textit{группой}, если для каждого элемента $a \in M$ найдется \textit{обратный элемент}, т.е.
		\begin{equation*} 
			\forall a \in M \ \exists a^{-1} \in M \suchthat a \circ a^{-1} = a^{-1} \circ a = e.
		\end{equation*} \n
		Группы обычно обозначаются $\langle M, \circ \rangle.$
	\end{definition}
	\begin{definition}
		Группа $G$ называется \textit{коммутативной} или \textit{абелевой}, если групповая операция \textit{коммутативна}, т.е.
		\begin{equation*}
			\forall a, b \in G{:} \ ab = ba.
		\end{equation*}
	\end{definition}
	\thispagestyle{empty}
	\begin{definition}
		\textit{Порядок} группы $G$ — это число элементов в $G$. Группа называется \textit{конечной}, если её порядок конечен, и \textit{бесконечной} иначе. \n
		Порядок группы $G$ обозначается $|G|$.
	\end{definition} 
	\newpage
	\begin{exmpls}
		\
		\begin{enumerate}
			\setlength\itemsep{0.1em}
			\item Числовые \textit{аддитивные} группы: \newline
			$\langle \integers, + \rangle, \, \langle \rationals, + \rangle, \, 
			\langle \real, + \rangle, \,
			\langle \complex, + \rangle, \, 
			\langle \integers_n, + \rangle.$ 
			\item Числовые \textit{мультипликативные} группы: \newline
			$\langle \rationals \, \backslash \, \{0\}, \times \rangle, \,
			\langle \real \, \backslash \, \{0\}, \times \rangle, \,
			\langle \complex \, \backslash \, \{0\}, \times \rangle, \,
			\langle \integers_p \, \backslash \, \{0\}, \times \rangle, \, p$~--- простое.
			\item Группы матриц: \newline
			${\GL_n(\real) = \{A \in \mathrm{Mat}_{n \times n}(\real) \ | \ \det A \neq 0 \}}$~--- \textit{полная линейная группа}; \newline
			${\SL_n(\real) = \{A \in \mathrm{Mat}_{n \times n}(\real) \ | \ \det A = 1 \}}$~--- \textit{специальная линейная
				группа}; \newline
			${\Orth_n(\real)} = \{A \in \mathrm{Mat}_{n \times n}(\real) \ | \ A \cdot A^T = I\}$~--- \textit{ортогональная группа}; \newline
			${\SOrth_n(\real) = \Orth_n(\real) \cap \SL_n(\real)}$~--- \textit{специальная ортогональная группа.}
			\item Группы перестановок: \newline
			\textit{симметрическая группа} $S_n$~--- все перестановки длины $n$;\newline
			\textit{знакопеременная группа} $A_n$~--- все чётные перестановки длины $n$.
			\item Группы преобразований подобия: гомотетии, движения (осевые и скользящие симметрии, параллельные переносы, повороты).
		\end{enumerate}
	\end{exmpls}
	\begin{definition}
		Для описания структур групп часто используются \textit{таблицы Кэли}. Они представляют собой квадратные таблицы, заполненные результатами применения бинарной операции к элементам множества.
	\end{definition}
	\begin{exmpl*}
		Таблица Кэли для группы $\langle \{1, 3, 5, 7\}, \times (\mod 8) \rangle{:}$
		\begin{center}
			\begin{tabular}{c|cccc}
				$\times$ & 1 & 3 & 5 & 7\\
				\hline
				1 & 1 & 3 & 5 & 7\\
				
				3 & 3 & 1 & 7 & 5\\
				
				5 & 5 & 7 & 1 & 3\\
				
				7 & 7 & 5 & 3 & 1\\
			\end{tabular}
		\end{center} \n
	\end{exmpl*}
	\newpage
	\section{Подгруппы. Описание всех подгрупп в группе $\langle \integers, + \rangle$. Циклические подгруппы и группы. Порядок элемента. Связь между порядком элемента и порядком порождаемой им циклической подгруппы}
	
	\begin{definition}	
		Подмножество $H$ группы $G$ называется \textit{подгруппой} и обозначается $H < G$, если выполнены следующие условия:
		\begin{enumerate}
			\setlength\itemsep{0.1em}
			\item $e \in H;$
			\item $\forall a,b \in H{:} \ ab \in H;$
			\item $\forall a \in H{:} \ a^{-1} \in H.$
		\end{enumerate}
	\end{definition}
	В каждой группе $G$ есть \textit{несобственные} или \textit{тривиальные} подгруппы $H = \{e\}$ и $H = G$. Все прочие подгруппы называются \textit{собственными}.
	\begin{exmpls}
		\
		\begin{enumerate}
			\setlength\itemsep{0.1em}
			\item ${\langle \integers < \rationals < \real < \complex, + \rangle}$
			\item ${\GL_n(\real) > \Orth_n(\real) > \SOrth_n(\real); \ \GL_n(\real) > \SL_n(\real).}$
			\item ${S_n > A_n.}$
		\end{enumerate}
	\end{exmpls}
	\begin{theorem*}[\textbf{Критерий подгруппы}]
		Пусть $G$~--- группа, тогда
		\begin{equation*}
			H < G \Longleftrightarrow \forall a,b \in H{:} \ a \circ b^{-1} \in H.
		\end{equation*}
	\end{theorem*}
	\begin{proof}
		Определим на $H$ вспомогательное отношение ${R_H = \{(a,b)\ | \ a \circ b^{-1} \in H\}}.$ Покажем, что $R_H$ является отношением эквивалентности. Для этого проверим, что оно ${\text{рефлексивно} \ (1)}$, симметрично $(2)$ и транзитивно $(3)$:
		\begin{enumerate}
			\setlength\itemsep{0.1em}
			\item $a \circ a^{-1} = e \in H;$
			\item $ab^{-1} \in H \Longrightarrow b a^{-1} = (ab^{-1})^{-1} \in H;$
			\item $ab^{-1} \in H, \ bc^{-1} \in H \Longrightarrow ac^{-1} = (ab^{-1})(bc^{-1}) = a(b^{-1}b)c^{-1} \in H.$
		\end{enumerate} \n
		
		Рефлексивность $R_H$ определяет наличие нейтрального элемента, симметричность~-- наличие обратного элемента, транзитивность~-- ассоциативность заданной бинарной операции. Каждый класс эквивалентности будет ассоциирован с некоторой подгруппой (как с алгебраически замкнутым множеством). 
	\end{proof}
	\begin{statement*}
		Всякая подгруппа в ${}\langle\integers, +\rangle$ имеет вид $k\integers$ для некоторого ${k \in \naturals_0 \ (\mathbb{N}_0 = \mathbb{N} \cup \{0\})}$.
	\end{statement*}
	\begin{proof}
		Очевидно, что все подмножества вида $k\mathbb{Z}$ являются подгруппами в $\integers$.
		Пусть ${H < \integers}$. Если ${H = \{0\}}$, то ${H = 0\integers}$.
		Иначе положим ${k = \min(H \cap N) \neq 0}$ (это множество непусто, т.к. ${\forall x \in H \cap N{:} ~-x \in H}$), тогда ${k\integers \subseteq H}$.
		Покажем, что ${k\integers = H}$. Пусть ${a \in H}$ — произвольный элемент. Поделим его на $k$ с остатком:
		\begin{equation*}
			a = qk + r, \text{где} \ k \in H, \  
			0 \leqslant r < k \Rightarrow r = a - qk \in H.
		\end{equation*}
		В силу выбора $k$ получаем: ${r = 0 \Rightarrow a = qk \in k\integers}$.
	\end{proof}
	\begin{definition}
		Пусть $G$~--- группа, $g \in G$ и $n \in \integers$. \textit{Степень} элемента $g$ определяется следующим образом:
		\begin{equation*}
			g^{n}=
			\begin{cases*}
				\underbrace{g \mathellipsis g}_n,  &$n > 0$ \\
				e,  &$n = 0$ \\
				\underbrace{g^{-1} \mathellipsis g^{-1}}_n, &$n < 0$
			\end{cases*}
		\end{equation*}
		и обладает свойствами:\n
		$\forall m, n \in \integers:$
		\begin{enumerate}
			\setlength\itemsep{0.1em}
			\item $g^m \cdot g^n = g^{m+n};$
			\item $(g^m)^{-1} = g^{-m};$
			\item $(g^m)^n = g^{mn}.$
		\end{enumerate}
	\end{definition}
	\newpage
	\begin{definition}
		Пусть $G$~--- группа и $g \in G$. \textit{Циклической подгруппой}, порожденной элементом $g$, называется подмножество ${\{g^n \ | \ n \in \integers\} \subseteq G}$. \n
		Циклическая подгруппа, порождённая элементом $g$, обозначается $\langle g \rangle$. Элемент $g$ называется \textit{порождающим} или \textit{образующим} для подгруппы $\langle g \rangle$.
	\end{definition}
	\begin{exmpl*}
		Подгруппа ${2\integers < \langle \integers, + \rangle}$ является циклической, и в качестве порождающего элемента в ней можно взять $g = 2$ или $g = -2.$ Другими словами, ${2\integers = \langle 2 \rangle = \langle -2 \rangle.}$
	\end{exmpl*}
	\begin{definition}
		Группа $G$ называется \textit{циклической}, если 
		\begin{equation*}
			\exists g \in G \suchthat G = \langle g \rangle.
		\end{equation*}
		Циклическая группа порядка $n$ обозначается ${C_n.}$
	\end{definition}
	\begin{exmpls}
		${\langle \integers, + \rangle; \ \langle \integers_n, + \rangle, n \geqslant 1.}$
	\end{exmpls}
	\begin{definition}
		Пусть $G$ — группа и ${g \in G}$. \textit{Порядком} элемента $g$ называется наименьшее ${m \in \naturals \suchthat g^m = e}$. Если такого натурального числа $m$ не существует, говорят, что порядок элемента $g$ равен бесконечности. Порядок элемента обозначается $\ord(g)$.
	\end{definition}
	\begin{remark}
		\begin{equation*}
			\ord(g) = 1 \Longleftrightarrow g = e.
		\end{equation*}
		
	\end{remark}
	\begin{statement*}
		Если $G$~--- группа и $g \in G$, то $\ord(g) = |\langle g \rangle|.$
	\end{statement*}
	\begin{proof}
		Заметим, что если $g^k = g^s$, то $g^{k - s} = e$. Поэтому если элемент $g$ имеет бесконечный порядок, то все элементы $g^n, n \in \integers$, попарно различны, и подгруппа $\langle g \rangle$ содержит бесконечно много элементов. 
		Если же ${\ord(g) = m}$, то из минимальности числа $m$ следует, что элементы ${e = g^0, g^1, g^2, \mathellipsis , g^{m-1}}$ попарно различны.
		Далее, ${\forall n \in \integers{:} \ n = mq + r, \ \text{где} \ 0 \leqslant r \leq m - 1, \ \text{и}}$
		\begin{equation*}
			g^n = g^{mq + r} = (g^m)^q g^r = e^q g^r = g^r.
		\end{equation*}
		Следовательно, $\langle g \rangle = \{e, g, g^2, \mathellipsis, g^{m - 1}\}$ и $|\langle g \rangle| = m.$
	\end{proof}
	Очевидно, что всякая циклическая группа коммутативна и не более чем счётна.
	\newpage
	\section{Смежные классы. Индекс подгруппы. Теорема Лагранжа}
	\setcounter{definition}{0}
	\begin{definition}
		\textit{Левым смежным классом} элемента $g$ группы $G$ по подгруппе $H$ называется подмножество
		\begin{equation*}
			gH = \{gh \ | \ h \in H\},
		\end{equation*}
		аналогично определяется \textit{правый смежный класс}:
		\begin{equation*}
			Hg = \{hg \ | \ h \in H\}.
		\end{equation*}
	\end{definition}
	\begin{lemma*}
		Пусть $G$~--- конечная подгруппа, тогда $\forall g \in G{:} \ |gH| = |H|.$
	\end{lemma*}
	\begin{proof}
		Поскольку ${gH = \{gh \ | \ h \in H\}}$, в $gH$ элементов не больше, чем в $H$. Если ${gh_1 = gh_2}$, то домножив слева на $g^{-1},$ получаем ${h_1 = h_2}$. Значит, все элементы вида $gh$, где ${h \in H}$, попарно различны, откуда ${|gH| = |H|}$.
	\end{proof}
	\begin{definition}
		Пусть $G$~--- группа, $H < G$. \textit{Индексом} подгруппы $H$ в группе $G$ называется число левых смежных классов $G$ по $H$.
	\end{definition} \n
	Индекс группы $G$ по подгруппе $H$ обозначается $[G : H]$.
	\begin{theorem*}[\textbf{Лагранж}]
		Пусть $G$~--- конечная группа, $H < G$. Тогда
		\begin{equation*}
			|G| = |H| \cdot [G : H].
		\end{equation*}
	\end{theorem*}
	\begin{proof}
		Каждый элемент группы $G$ лежит в (своём) левом смежном классе по подгруппе $H$, разные смежные классы не пересекаются (по следствию из доказательства критерия подгруппы) и каждый из них содержит по $|H|$ элементов (по предыдущей лемме).
	\end{proof}
	\newpage
	\begin{consequence}
		$|G| \divby |H|$.
	\end{consequence}
	\begin{consequence}
		$|G| \divby \ord(g)$.
	\end{consequence}
	\begin{proof}
		Вытекает из следствия 1 и того, что $\ord(g) = |\langle g \rangle|$.
	\end{proof}
	\begin{consequence}
		$g^{|G|} = e$.
	\end{consequence}
	\begin{proof}
		Из предыдущего следствия получаем: ${|G| = \ord(g) \cdot s, \ s \in \naturals} \Longrightarrow g^{|G|} = (g^{\ord(g)})^s = e^s = e.$
	\end{proof}
	\begin{consequence}[\textbf{Малая теорема Ферма}]
		Пусть $\overline{a}$~--- ненулевой вычет по простому модулю $p$, тогда $\overline{a}^{p-1} \equiv 1 \ (\mod \, p).$
	\end{consequence}
	\begin{proof}
		Достаточно применить следствие 3 к группе $\langle \integers_p \ \backslash \ \{0\}, \times \rangle.$
	\end{proof}
	\begin{consequence}
		Пусть $|G|$~--- простое число, тогда $G$~--- циклическая группа, порождённая любым своим ненейтральным элементом.
	\end{consequence}
	\begin{proof}
		Пусть $g \in G$~--- произвольный ненейтральный элемент. Тогда циклическая подгруппа $\langle g \rangle$ содержит более одного элемента и $|\langle g \rangle|$ делит $|G|$ по следствию 1. Значит, $|\langle g \rangle| = |G|$, откуда $G = \langle g \rangle$. 
	\end{proof}
	\newpage
	\section{Метрические пространства. Изометрии и движения. Группы движений. Диэдральные группы}
	\setcounter{definition}{0}
	\begin{definition}
		Упорядоченная пара ${(M, d),}$ состоящая из множества $M$ и отображения ${d{:} \ M \times M \rightarrow \real,}$  называется \textit{метрическим пространством}, если ${\forall x, y \in M{:}}$
		\begin{enumerate}
			\setlength\itemsep{0.1em}
			\item ${d(x, y) = 0 \Leftrightarrow x = y}$ (\textit{аксиома тождества});
			\item ${d(x, y) \geqslant 0}$ (\textit{аксиома неотрицательности});
			\item ${d(x, y) = d(y, x)}$ (\textit{аксиома симметричности});
			\item ${d(x, y) + d(y, z) \geqslant d(x, y)}$ (\textit{аксиома} или \textit{неравенство треугольника}).
		\end{enumerate}
	\end{definition}
	\begin{definition}
		Пусть $X$ и $Y$~--- метрические пространства. Отображение ${f{:} \ X \rightarrow Y}$ называется \textit{изометрией}, если оно сохраняет расстояние между точками:
		\begin{equation*}
			\forall x, x' \in X{:} \ |f(x) - f(x')|_Y = |x - x'|_X,
		\end{equation*} \n
		где ${|a - b|_S}$~--- расстояние между $a$ и $b$ в пространстве $S$.\n
		Если ${X = Y,}$ $f$ называют \textit{движением}.
	\end{definition}
	\begin{definition}
		Движение называют \textit{собственным}, если оно сохраняет \textit{ориентацию} пространства.
	\end{definition}
	\begin{definition}
		Пусть $E$~--- \textit{евклидово аффинное пространство} и ${F \subseteq E}$~--- геометрическая фигура. \textit{Группой движений (изометрий)} ${\Isom(F)}$ фигуры $F$ называется множество тех движений аффинного пространства $E$, которые  переводят фигуру $F$ в себя:
		\begin{equation*}
			\Isom(F) = \{\phi{:} \ E \rightarrow E \ | \ \phi\textrm{~--- движение}, \ \phi(F) = F\}.
		\end{equation*} 
		В качестве групповой операции рассматривается операция композиции движений.
	\end{definition}
	\begin{remark}
		Группа собственных движений ${\Isom(F)^+}$ является подгруппой группы движений ${\Isom(F)}$ фигуры $F.$
	\end{remark}
	\newpage
	\begin{definition}
		Группа движений правильного $n$-угольника ${\Delta_n \subset \real^2}$ называется \textit{диэдральной группой} $D_n{:}$
		\begin{equation*}
			D_n = \Isom(\Delta_n).
		\end{equation*} \n
		Есть всего 2 вида таких преобразований:
		\begin{enumerate}
			\setlength\itemsep{0.1em}
			\item $n$ вращений относительно центра на угол, кратный $\frac{2\pi}{n}$ (вращение на угол $\phi$ обозначается $\Rot_\phi$);
			\item $n$ симметрий относительно осей симметрии (симметрия относительно прямой $l$ обозначается $\Sym_l$). \n В случае нечетного $n$ любая ось симметрии проходит через центр $\Delta_n$ и одну из вершин, в случае четного $n$ любая ось симметрии проходит либо через противоположные вершины, либо через середины противоположных сторон. 
		\end{enumerate}
	\end{definition}
	\begin{statement*}
		${|D_n| = 2n.}$
	\end{statement*}
	\begin{remark}
		Группа собственных движений $\Delta_n$ содержит только повороты:
		\begin{equation*}
			\Isom(D_n)^+ = \{\Rot_{\frac{2\pi k}{n}}\}.
		\end{equation*}
	\end{remark}
	\begin{exmpl*}
		Таблица Кэли группы $D_4$ квадрата $ABCD{:}$
		\begin{center}
			\begin{tabular}{c|c|c|c|c|c|c|c|c}
				$\circ$ & $\id$ & $\Rot_{\frac{\pi}{2}}$ & $\Rot_{\pi}$ & $\Rot_{\frac{3\pi}{2}}$ & 
				$\Sym_h$ &
				$\Sym_v$ &
				$\Sym_{AC}$ &
				$\Sym_{BD}$ \\
				\hline
				$\id$ & $\id$ & $\Rot_{\frac{\pi}{2}}$ & $\Rot_{\pi}$ & $\Rot_{\frac{3\pi}{2}}$ & 
				$\Sym_h$ &
				$\Sym_v$ &
				$\Sym_{AC}$ &
				$\Sym_{BD}$ \\
				\hline
				$\Rot_{\frac{\pi}{2}}$ & $\Rot_{\frac{\pi}{2}}$ & $\Rot_{\pi}$ & $\Rot_\frac{3\pi}{2}$ & $\id$ & $\Sym_{BD}$ & $\Sym_{AC}$ & $\Sym_h$ & $\Sym_v$ \\
				\hline
				$\Rot_{\pi}$ & $\Rot_{\pi}$ & $\Rot_\frac{3\pi}{2}$ & $\id$ & $\Rot_\frac{\pi}{2}$ & $\Sym_v$ & $\Sym_h$ & $\Sym_{BD}$ & $\Sym_{AC}$ \\ \hline
				$\Rot_\frac{3\pi}{2}$ & $\Rot_\frac{3\pi}{2}$ & $\id$ & $\Rot_\frac{\pi}{2}$ & $\Rot_{\pi}$ & $\Sym_{AC}$ & $\Sym_{BD}$ & $\Sym_v$ & $\Sym_h$ \\
				\hline
				$\Sym_h$ & $\Sym_h$ & $\Sym_{AC}$ & $\Sym_v$ & $\Sym_{BD}$ & $\id$ & $\Rot_{\pi}$ & $\Rot_\frac{\pi}{2}$ & $\Rot_\frac{3\pi}{2}$
				\\
				\hline
				$\Sym_v$ & $\Sym_v$ & $\Sym_{BD}$ & $\Sym_h$ & $\Sym_{AC}$ & $\Rot_{\pi}$ & $\id$ & $\Rot_\frac{3\pi}{2}$ & $\Rot_\frac{\pi}{2}$
				\\
				\hline
				$\Sym_{AC}$ & $\Sym_{AC}$ & $\Sym_v$ & $\Sym_{BD}$ & $\Sym_h$ & $\Rot_\frac{3\pi}{2}$ & $\Rot_\frac{\pi}{2}$ & $\id$ & $\Rot_{\pi}$
				\\
				\hline
				$\Sym_{BD}$ & $\Sym_{BD}$ & $\Sym_h$ & $\Sym_{AC}$ & $\Sym_v$ & $\Rot_\frac{\pi}{2}$ & $\Rot_\frac{3\pi}{2}$ & $\Rot_{\pi}$ & $\id$\\
			\end{tabular}
		\end{center} \n
	\end{exmpl*}
	\newpage
	\section{Группа перестановок. Цикловое разложение. Порядок элементов в $S_n$. Теорема Кэли}
	\setcounter{definition}{0}
	\begin{definition}
		Пусть задано множество ${X = \{1, 2, \mathellipsis, n\}, \ n \in \naturals}$. Множество всех возможных биекций ${\Pi =\{\pi_i{:} \ X \leftrightarrow X\}}$ с операцией композиции $\circ$ образует группу ${S_n = \langle \Pi, \circ \rangle,}$ называемую \textit{симметрической группой} или \textit{группой перестановок}.
	\end{definition}
	\begin{statement}
		\begin{equation*}
			|S_n| = \mathrm{Card} \, \Pi = n!.
		\end{equation*}
	\end{statement}
	\begin{proof}
		Символ 1 можно подходящей перестановкой $\sigma$ перевести в любой другой символ ${\sigma(1)}$, для чего существует в точности $n$ различных возможностей. Но зафиксировав ${\sigma(1)}$, в качестве ${\sigma(2)}$ можно брать лишь один из оставшихся ${n - 1}$ символов и т.д. Всего возможностей выбора ${\sigma(1), \sigma(2), \mathellipsis, \sigma(n)}$, значит и всех перестановок будет ${n(n - 1) \mathellipsis 2 \cdot 1 = n!.}$
	\end{proof}
	\begin{statement}
		Любая перестановка может быть представлена в виде композиции непересекающихся циклов.
	\end{statement}
	\begin{definition}
		Цикл длины 2 называется \textit{транспозицией}.
	\end{definition}
	\begin{statement}
		Каждая перестановка ${\pi \in S_n}$ является композицией транспозиций.
	\end{statement}
	\begin{statement}
		Непересекающиеся циклы коммутируют.
	\end{statement}
	\begin{statement}
		Порядок цикла равен его длине.
	\end{statement}
	\begin{statement}
		Порядок перестановки равен $\NOK$ длин циклов в его цикловом разложении.
	\end{statement}
	\begin{definition}
		\textit{Цикловой структурой} перестановки ${\pi \in S_n}$ называется упорядоченный набор чисел ${\CS(\pi) = (c_1, c_2, \mathellipsis, c_n),}$ где $c_i$~--- количество циклов длины $i$ в разложении $\pi.$
	\end{definition}
	\newpage
	\begin{exmpl*}
		Пусть $G = S_3, \ H = \langle(12)\rangle = \{\id, (12)\}.$ Найдём все левые и правые смежные классы $G$ по $H$ (произвольный элемент обозначим $a$):
		\begin{center}
			\begin{tabular}{c|c|c}
				$a$ & $aH$ & $Ha$\\
				\hline
				$\id$ & $aH$ & $Ha$ \\
				\hline
				$(12)$ & $\{(12), \id\}$ & $\{(12), \id\}$ \\
				\hline
				$(13)$ & $\{(13), (123)\}$ & $\{(13), (132)\}$ \\
				\hline
				$(23)$ & $\{(23), (132)\}$ & $\{(23), (123)\}$ \\
				\hline
				$(123)$ & $\{(123), (13)\}$ & $\{(123), (23)\}$ \\
				\hline
				$(132)$ & $\{(132), (23)\}$ & $\{(132), (13)\}$
			\end{tabular}
		\end{center} \n
	\end{exmpl*}
	\begin{theorem*}[\textbf{Кэли}]
		Любая конечная группа $G$ порядка $n$ изоморфна некоторой подгруппе симметрической группы $S_n{:}$
		\begin{equation*}
			\forall a, g \in G{:} \ a \mapsto \pi_a, \quad \pi_a(g) = a \circ g.
		\end{equation*} \n
	\end{theorem*}
	\begin{proof}
		содержимое...
	\end{proof}
	\newpage
	\section{Нормальные подгруппы. Факторгруппы}
	\setcounter{definition}{0}
	\begin{definition}
		Подгруппа $H$ группы $G$ называется \textit{нормальной}, если 
		\begin{equation*}
			\forall g \in G{:} \ gH = Hg. 
		\end{equation*}
		\n
		Обозначается $H \lhd G.$
	\end{definition}
	\begin{statement*}
		Пусть $H$~--- подгруппа группы $G$, тогда следующие условия эквивалентны:
		\begin{enumerate}
			\setlength\itemsep{0.1em}
			\item $H$ нормальна;
			\item $\forall g \in G{:} \ gHg^{-1} = H$;
			\item $\forall g \in G{:} \ gHg^{-1} \subseteq H$.
		\end{enumerate}
	\end{statement*}
	\begin{proof}
		\
		\begin{itemize}
			\setlength\itemsep{0.1em}
			\item[(1)] $\Longrightarrow (2){:} \ gH = Hg \ | \cdot g^{-1} \Longrightarrow gHg^{-1} = H$.
			\item[(2)] $\Longrightarrow (3){:}$ очевидно.
			\item[(3)] ${\Longrightarrow (2){:} \ gHg^{-1} \subseteq H \Longrightarrow gHg^{-1} \subseteq H \ | \cdot g \Longrightarrow gH \subseteq Hg.}$
			\n
			${\text{Если} \ g = g^{-1}, \ \text{то} \ g \cdot | \ g^{-1}Hg \subseteq H \Longrightarrow Hg \subseteq gH \Longrightarrow gH = Hg. \qedhere}$ 
		\end{itemize}
	\end{proof}
	Рассмотрим множество смежных классов по нормальной подгруппе, обозначенной $G/H$.
	Определим на $G/H$ бинарную операцию, полагая, что $(g_1H)(g_2H) = (g_1g_2)H$.
	\n
	Пусть ${g'_1H = g_1H}$ и ${g'_2H=g_2H,}$ тогда ${g'_1 = g_1h_1, \ g'_2=g_2h_2,}$ ${\text{где} \ h_1, h_2 \in H.}$
	\begin{equation*}
		\begin{gathered}
			(g'_1H)(g'_2H) =
			(g'_1g'_2)H = 
			(g_1h_1g_2h_2)H = 
			(g_1g_2 \underbrace{g^{-1}_2h_1g_2}_{\in H}h_2)H \subseteq (g_1g_2)H \Longrightarrow \\
			\Longrightarrow (g'_1g'_2)H = (g_1g_2)H.
		\end{gathered}
	\end{equation*}
	\newpage
	\begin{statement*}
		$G/H$ является группой.
	\end{statement*}
	\begin{proof}
		Проверим аксиомы группы:
		\begin{enumerate}
			\setlength\itemsep{0.1em}
			\item Ассоциативность очевидна.
			\item Нейтральный элемент~--- $eH.$
			\item Обратный к $gH$~--- $g^{-1}H.$ \qedhere 
		\end{enumerate}
	\end{proof}
	\begin{definition}
		Множество $G/H$ с указанной операцией называется \textit{факторгруппой} группы $G$ по нормальной подгруппе $H$.
	\end{definition}
	\begin{exmpl*}
		Если $G = \langle \integers, + \rangle$ и $H = n\integers$, то $G/H$~--- группа вычетов $\langle \integers_n, + \rangle.$
	\end{exmpl*}
	\newpage	
	\section{Гомоморфизмы групп, их виды. Свойства гомоморфизмов. Четверная группа Клейна. Ядро и образ гомоморфизма}
	\setcounter{definition}{0}
	\begin{definition}
		Пусть $\langle G, \circ \rangle$ и $\langle F, * \rangle$~--- группы.
		\n
		Отображение $\phi{:} \ G \rightarrow F$ называется \textit{гомоморфизмом}, если
		\begin{equation*}
			\forall g_1, g_2 \in G{:} \ \phi(g_1 \circ g_2) = \phi(g_1) * \phi(g_2).
		\end{equation*}
	\end{definition}
	\begin{remark}
		Пусть ${\phi{:} \ G \rightarrow F}$~--- гомоморфизм групп, и пусть ${e_G, \ e_F}$~--- нейтральные элементы групп $G$ и $F$ соответственно, тогда:
		\begin{enumerate}
			\setlength\itemsep{0.1em}
			\item $\phi(e_G) = e_F$
			\item $\forall g \in G{:} \ \phi(g^{-1}) = \phi(g)^{-1}$
		\end{enumerate}
	\end{remark}
	\begin{proof}
		\
		\begin{enumerate}
			\setlength\itemsep{0.1em}
			\item $\phi(e_G) = \phi(e_Ge_G) = \phi(e_G)\phi(e_G).$ \n
			Домножив обе крайние части равенства на $\phi(e_G)^{-1},$ получим
			${e_F = \phi(e_G).}$
			\item $\phi(g * g^{-1}) = e_F = \phi(g)\phi(g^{-1}).$ \n
			Умножив обе части на $\phi(g)^{-1},$ получаем необходимое. \qedhere
		\end{enumerate}
	\end{proof}
	\begin{definition}
		Гомоморфизм групп ${\phi{:} \ G \rightarrow F}$ называется 
		\begin{itemize}
			\item [~--] \textit{эндоморфизмом}, если $F = G$;
			\item [~--] \textit{мономорфизмом}, если $\phi$ инъективно;
			\item [~--] \textit{эпиморфизмом}, если $\phi$ сюръективно;
			\item [~--] \textit{изоморфизмом}, если $\phi$ биективно;
			\item [~--] \textit{автоморфизмом}, если $\phi$ является эндоморфизмом и изоморфизмом.
		\end{itemize} \n
		Группы $G$ и $F$ называются \textit{изоморфными}, если между ними существует изоморфизм. Обозначается: $G \cong F.$
	\end{definition}
	\newpage
	\begin{exmpl*}
		\textit{Четверная группа Клейна}~--- ациклическая коммутативная группа четвёртого порядка, задающаяся следующей таблицей Кэли: \n
		
		\begin{center}
			\begin{tabular}{c |c c c c}
				$\cdot$ & $e$ & $a$ & $b$ & $ab$\\
				\hline
				$e$ & $e$ & $a$ & $b$ & $ab$\\
				
				$a$ & $a$ & $e$ & $ab$ & $b$\\
				
				$b$ & $b$ & $ab$ & $e$ & $a$\\
				
				$ab$ & $ab$ & $b$ & $a$ & $e$\\
			\end{tabular}
		\end{center} \n
		Порядок каждого элемента, отличного от нейтрального, равен 2. \n
		Обозначается $V$ или $V_4$ (от нем. \textit{Vierergruppe} — четверная группа). \n
		Любая группа четвёртого порядка изоморфна либо циклической группе, либо четверной группе Клейна, наименьшей по порядку нециклической группе. Симметрическая группа $S_4$ имеет лишь две нетривиальные нормальные подгруппы~--- знакопеременную группу $A_4$ и четверную группу Клейна $V_4$, состоящую из перестановок ${\mathrm{id}, (12)(34), (13)(24), (14)(23).}$ \n
		Несколько примеров изоморфных ей групп:
		\begin{itemize}
			\setlength\itemsep{0.1em}
			\item[~--] \textit{прямая сумма} $\integers_2 \oplus \integers_2$;
			\item[~--] диэдральная группа $D_2;$
			\item[~--] множество ${\{0, 1, 2, 3\}}$ с операцией XOR;
			\item[~--] группа симметрий ромба $ABCD$ в трёхмерном пространстве, состоящая из 4 преобразований: ${\id, \Rot_\pi, \Sym_{AC}, \Sym_{BD}}$;
			\item[~--] группа поворотов тетраэдра на угол $\pi$ вокруг всех трёх рёберных медиан (вместе с тождественным поворотом).
		\end{itemize}
	\end{exmpl*}
	\newpage
	\begin{definition}
		\textit{Ядром} гомоморфизма ${\phi{:} \ G \rightarrow F}$ называется множество всех элементов $G$, которые отображаются в нейтральный элемент $F$, т.е.
		\begin{equation*}
			\ker \phi = \{g \in G \ | \ \phi(g) = e_F\}.
		\end{equation*} \n
		\textit{Образ} $\phi$ определяется как
		\begin{equation*}
			\im \phi = \phi(G) = \{f \in F \ | \ \exists g \in G \suchthat \phi(g) = f\}.
		\end{equation*}
	\end{definition}
	Очевидно, что $\ker \phi < G$ и Im$\, \phi < F.$
	\begin{lemma*}
		Гомоморфизм групп ${\phi{:} \ G \rightarrow F}$ инъективен тогда и только тогда, когда  $\ker \phi = \{e_G\}$.
	\end{lemma*}
	\begin{proof}
		Ясно, что если $\phi$ инъективен, то $\ker \phi = \{e_G\}.$ \n
		Обратно, пусть ${g_1, g_2 \in G}$ и ${\phi(g_1) = \phi(g_2).}$ Тогда ${g^{-1}_1g_2 \in \ker \phi,}$ поскольку ${\phi(g^{-1}_1g_2) = \phi(g^{-1}_1) \phi(g_2) = \phi(g_1)^{-1} \phi(g_2) = e_F.}$ Отсюда $g^{-1}_1g_2 = e_G$ и $g_1 = g_2.$
	\end{proof}
	\begin{consequence*}
		Гомоморфизм групп ${\phi{:} \ G \rightarrow F}$ является изоморфизмом тогда и только тогда, когда ${\ker \phi = \{e_G\}}$ и ${\im \phi = F.}$
	\end{consequence*}
	\begin{statement*}
		Пусть ${\phi{:} \ G \rightarrow F}$ гомоморфизм групп, тогда $\ker \phi \lhd G.$
	\end{statement*}
	\begin{proof}
		Достаточно проверить, что
		\begin{equation*}
			\forall g \in G \ \ \forall h \in \ker \phi{:} \ g^{-1}hg \in \ker \phi.
		\end{equation*}
		Это следует из цепочки равенств:
		\begin{equation*}
			\phi(g^{-1}_1hg) = \phi(g^{-1}) \phi(h) \phi(g) = \phi(g^{-1}) e_F \phi(g) = \phi(g^{-1}) \phi(g) = \phi(g)^{-1} \phi(g) = e_F.
		\end{equation*}
	\end{proof}
	\newpage
	\section{Теорема о гомоморфизме. Классификация циклических групп}
	\begin{theorem*}[\textbf{О гомоморфизме}]
		Пусть ${\phi{:} \ G \rightarrow F}$~--- гомоморфизм групп, тогда
		\begin{equation*}
			\im \phi \cong G/ \ker \phi.
		\end{equation*}
	\end{theorem*}
	\begin{proof}
		Рассмотрим отображение ${\psi{:} \ G/ \ker \phi \rightarrow \im \phi,}$ заданное формулой ${\psi(g \ker \phi) = \phi(g).}$ \n
		Достаточно проверить определение изоморфизма для $\psi.$ Для этого покажем, что заданное отображение корректно определено, биективно и гомоморфно.
		\begin{enumerate}
			\setlength\itemsep{0.1em}
			\item Проверим корректность $\psi{:}$ \n
			$\exists h_1, h_2 \in \ker \phi \suchthat g_1 \ker \phi = g_2 \ker \phi \Longrightarrow g_1 h_1 = g_2 h_2;$ \n
			$\psi(g_1 \ker \phi) = \phi(g_1) = \phi(g_1 h_1) = \phi(g_2 h_2) = \phi(g_2) = \psi(g_2 \ker \phi).$
			\item Докажем, что $\psi$~--- гомоморфизм: \n
			${\psi((g_1 \ker \phi)(g_2 \ker \phi)) = \psi ((g_1g_2) \ker \phi) = \phi(g_1 g_2) = \phi(g_1) \phi(g_2) =}$ \n
			${=\psi(g_1 \ker \phi) \psi(g_2 \ker \phi).}$
			\item Сюръективность видна из построения.
			\item Инъективность: \n
			${\psi(g_1 \ker \phi) = \psi(g_2 \ker \phi) \Longrightarrow \phi(g_1) = \phi(g_2) \Longrightarrow \phi(g_1) \phi(g_2)^{-1} = e_F \Longrightarrow}$ \n $\Longrightarrow \phi(g_1g_2^{-1}) = e_F \Longrightarrow g_1g_2^{-1} \in \ker \phi \Longrightarrow g_1 \ker \phi = g_2 \ker \phi. \qedhere$
		\end{enumerate} 
	\end{proof}
	\begin{exmpl*}
		Пусть ${G = \langle \real, + \rangle}$ и ${H = \langle \integers, + \rangle}$. Рассмотрим группу ${F = \langle \complex \, \backslash \, \{0\}, \times \rangle}$ и гомоморфизм
		\begin{equation*}
			\phi{:} \ G \rightarrow F, \quad g \mapsto e^{2\pi i g} = \cos(2\pi g) + i \sin(2\pi g).
		\end{equation*}
		Тогда ${\ker \phi = H}$ и факторгруппа $G/H$ изоморфна окружности $S^1$, рассматриваемой как подгруппа в $F$, состоящей из комплексных чисел с модулем равным 1.
	\end{exmpl*}
	\newpage
	\begin{theorem*}[\textbf{О классификации циклических групп}]
		Пусть $G$~--- циклическая группа. Тогда
		\begin{enumerate}
			\setlength\itemsep{0.1em}
			\item Если $|G| = \infty$, то $G \cong \langle \integers, + \rangle.$
			\item Если $|G| = n < \infty$, то $G \cong \langle \integers_n, + \rangle$.
		\end{enumerate}
	\end{theorem*}
	\begin{proof}
		Пусть ${G = \langle g \rangle.}$ Рассмотрим отображение $\phi{:} \ \integers \rightarrow G, \quad k \mapsto g^k.$
		\begin{equation*}
			\phi(k + l) = g^{k+l} = g^kg^l =
			\phi(k) \phi(l), \text{ поэтому } \phi \text{~--- гомоморфизм.}
		\end{equation*} \n
		Из определения циклической группы следует, что $\phi$ сюръективен, т.е. $\im \phi = G$. По теореме о гомоморфизме получаем ${G \cong \integers/\ker \phi,}$ т.к. ${\ker \phi < \integers \Longrightarrow \exists m \geqslant 0 \suchthat \ker \phi = m \integers}$ (любая подгруппа $\integers$ имеет вид $k\integers$). Если ${m = 0},$ то ${\ker \phi = \{0\},}$ откуда ${G \cong \integers \, / \, \{0\} \cong \integers.}$ Если ${m > 0,}$ то ${G \cong \integers/m\integers = \integers_m.}$
	\end{proof}
	
	\newpage
	\section{Прямое произведение групп. Разложение конечной циклической группы. Теорема о строении конечных абелевых групп}
	\setcounter{definition}{0}
	\begin{definition}
		\textit{Прямым произведением групп} $G_1, \mathellipsis, G_m$ называется группа
		\begin{equation*}
			G_1 \times \mathellipsis \times G_m = \{(g_1, \mathellipsis, g_m) \ | \ g_1 \in G_1, \mathellipsis, g_m \in G_m\}
		\end{equation*}
		с операцией $(g_1, \mathellipsis, g_m)(g'_1, \mathellipsis, g'_m) = (g_1g'_1, \mathellipsis g_mg'_m).$ \n
		Ясно, что эта операция ассоциативна, обладает нейтральным элементом $(e_{G_1}, \mathellipsis, e_{G_m})$ и для каждого элемента $(g_1, \mathellipsis, g_m)$ есть обратный элемент $(g^{-1}_1, \mathellipsis, g^{-1}_1).$
	\end{definition}
	\begin{remark}
		Группа $G_1 \times \mathellipsis \times G_m$ коммутативна тогда и только тогда, когда коммутативна каждая из групп $G_1, \mathellipsis, G_m.$
	\end{remark}
	\begin{remark}
		Если все группы ${G_1, \mathellipsis, G_m}$ конечны, то ${|G_1 \times \mathellipsis \times G_m| = |G_1| \mathellipsis |G_m|.}$
	\end{remark}
	\begin{definition}
		Говорят, что группа $G$ \textit{раскладывается в прямое произведение} своих подгрупп ${H_1, \mathellipsis, H_m}$, если отображение
		${H_1 \times \mathellipsis \times H_m \rightarrow G, \quad (h_1, \mathellipsis, h_m) \mapsto h_1 \mathellipsis h_m}$ является изоморфизмом.
	\end{definition}
	\begin{theorem*}
		Пусть ${n = pq}$~--- разложение натурального числа $n$ на два взаимно простых сомножителя. Тогда имеет место изоморфизм групп
		\begin{equation*}
			\integers_n \cong \integers_p \times \integers_q.
		\end{equation*}
	\end{theorem*}
	\newpage
	\begin{proof}
		Рассмотрим отображение
		\begin{equation*}
			\phi{:} \ \integers_n \rightarrow \integers_p \times \integers_q, \quad \phi(a \ \mod n) = (a \ \mod p, \, a \ \mod q).
		\end{equation*}
		\begin{enumerate}
			\setlength\itemsep{0.1em}
			\item Корректность следует из того, что $n \divby p, \ n \divby q.$
			\item $\phi$~--- гомоморфизм, т.к.
			\begin{equation*}
				\phi((a + b) \ \mod n) = \phi(a \ \mod n) + \phi(b \ \mod n).
			\end{equation*}
			\item $\phi$ инъективен: \n
			Если ${\phi(a \ \mod n) =(0, \, 0),}$ то ${a \divby p, \ a \divby q.}$ Но так как ${\NOK(p, \, q) = 1,}$ \n
			получаем, что $a \, | \, n.$ Тогда ${a \equiv 0 \ (\mod n),}$ т.е. ${\ker \phi = \{0\}.}$
			\item $\phi$ сюръективен, т.к. ${|\integers_n| = n = p \cdot q = |\integers_p \times \integers_q|.}$ \qedhere
		\end{enumerate}
	\end{proof}
	\begin{consequence*}
		Пусть ${n \geqslant 2}$~--- натуральное число и ${n = p^{k_1}_1 \mathellipsis p^{k_s}_s}$~--- его разложение в произведение простых множителей ${(p_i \neq p_j \text{ при } i \neq j).}$ Тогда имеет место изоморфизм групп
		\begin{equation*}
			\integers_n \cong \integers_{p^{k_1}_1} \times \mathellipsis \times \integers_{p^{k_s}_s}.
		\end{equation*}
	\end{consequence*}
	\begin{definition}
		Конечная абелева группа $A$ называется \textit{примарной}, если ${|A| = p^k,}$ где $p$~--- простое и ${k \in \naturals.}$
	\end{definition}
	\begin{theorem*}[\textbf{О строении конечных абелевых групп}]
		Пусть $A$~--- конечная абелева группа. Тогда ${A \cong \integers_{p^{k_1}_1} \times \mathellipsis \times \integers_{p^{k_t}_t},}$ где ${p_1, \mathellipsis, p_t}$~--- простые числа (не обязательно различные) и ${k_1, \mathellipsis, k_t \in \naturals.}$ Более того, набор примарных циклических множителей ${\integers_{p^{k_1}_1}, \mathellipsis, \integers_{p^{k_t}_t}}$ определен однозначно с точностью до перестановки (в частности, число этих множителей определено однозначно).
	\end{theorem*}
	\newpage
	\section{Экспонента конечной абелевой группы и критерий цикличности}
	\begin{definition*}
		\textit{Экспонентой} конечной абелевой группы $A$ называется число
		\begin{equation*}
			\exp A =\min \{m \in \naturals \ | \ \forall a \in A{:} \ ma = 0\}.
		\end{equation*}
	\end{definition*}
	\begin{remark}
		\
		\begin{enumerate}
			\setlength\itemsep{0.1em}
			\item Из того, что ${\forall a \in A \  \forall m \in \integers{:} \ ma = 0 \Longleftrightarrow m \divby \ord(a),}$ определение экспоненты можно переписать в виде ${\exp A = \NOK \{\ord(a) \ | \ a \in A\}.}$
			\item Из того, что ${\forall a \in A{:} \ |A| \divby \ord(a),}$ следует, что $|A|$~--- общее кратное
			множества ${\{\ord(a) \ | \ a \in A\},}$ а значит, ${|A| \divby \exp A.}$ В частности,
			${\exp A \leqslant |A|.}$
		\end{enumerate}
	\end{remark}
	\begin{theorem*}[\textbf{Критерий цикличности}]
		Группа $A$ является циклической тогда и только тогда, когда ${\exp A = |A|.}$
	\end{theorem*}
	\begin{proof}
		Пусть ${|A| = n = p^{k_1}_1 \cdot \mathellipsis \cdot p^{k_s}_s}$~--- разложение на простые множители, где $p_i$~--- простое и ${k_s \in \naturals \ (p_i \neq p_j \text{ при } i \neq j).}$ \newline
		{\textit{Необходимость.}} Если ${A = \langle a \rangle,}$ то ${\ord (a) = n,}$ откуда ${\exp A = n.}$ \newline
		{\textit{Достаточность.}} Если ${\exp A = n,}$ то для ${i = 1, \mathellipsis, s \quad \exists c_i \in A \suchthat \ord(c_i) = p^{k_i}_i m_i, \ m_i \in \naturals.}$ Для каждого ${i = 1, \mathellipsis, s}$ положим ${a_i = m_i c_i,}$ тогда ${\ord(a_i) = p^{k_i}_i.}$ Рассмотрим элемент ${a = a_1 + \mathellipsis + a_s}$ и покажем, что ${\ord(a) = n.}$ Пусть ${\exists m \in \naturals \suchthat ma = 0,}$ т.е. ${ma_1 + \mathellipsis + ma_s = 0.}$ При фиксированном ${i \in \{1, \mathellipsis, s\}}$ умножим обе части последнего равенства на ${n_i = n/p^{k_i}_i.}$ Видно, что ${\forall i \neq j{:} \ mn_ia_j = 0}$, поэтому в левой части останется только слагаемое ${mn_ia_i,}$ откуда ${mn_ia_i = 0 \Longrightarrow mn_i \divby p^{k_i}_i,}$ а т.к. 
		${n_i \ndivby p_i}$, то ${m \divby p^{k_i}_i.}$  В силу произвольности выбора $i$ отсюда вытекает, что ${m \divby n,}$ и т.к. ${na = 0,}$ то окончательно получаем ${\ord(a) = n.}$ Значит, ${A = \langle a \rangle}$~--- циклическая группа.
	\end{proof}
	\newpage
	\section{Кольца. Коммутативные кольца. Обратимые элементы, делители нуля и нильпотенты. Поля. Критерий того, что кольцо вычетов является полем}
	\setcounter{definition}{0}
	\begin{definition}
		\textit{Кольцо}~--- это множество $R,$ на котором заданы две бинарные операции ${\flqq+\frqq}$ (сложение) и ${\flqq \cdot \frqq}$ (умножение), удовлетворяющее следующим условиям:
		\begin{enumerate}
			\setlength\itemsep{0.1em}
			\item $\langle R, + \rangle$ ~--- абелева группа;
			\item $(R, \cdot)$~--- алгебраическая структура;
			\item $\forall a, b, c \in R{:} \ a(b + c) = ab + ac$ и $(a + b)c = ac + bc.$
			%\item $\forall a, b, c \in R{:} \ (ab)c = a(bc);$
			%\item $\exists 1 \in R \suchthat 1 \cdot a = a \cdot 1 = a \quad \forall a \in R.$
		\end{enumerate} 
	\end{definition}
	\begin{remark}
		\
		\begin{enumerate}
			\setlength\itemsep{0.1em}
			\item $\forall a \in R{:} \ 0 \cdot a = a \cdot 0 = 0;$
			\item Если $|R| > 1,$ то $1 \neq 0.$ 
		\end{enumerate}
	\end{remark}
	\begin{proof}
		\
		\begin{enumerate}
			\setlength\itemsep{0.1em}
			\item $a0 = a(0 + 0) = a0 + a0 \Longrightarrow 0 = a0.$
			\item Следует из условий выше. \qedhere
		\end{enumerate}
	\end{proof}
	\begin{exmpls}
		\
		\begin{enumerate}
			\setlength\itemsep{0.1em}
			\item Числовые кольца $\integers, \rationals, \real, \complex;$
			\item Кольцо $\integers_n$ вычетов по модулю $n;$
			\item Кольцо матриц $\mathrm{Mat}_{n \times n}(\real);$
			\item Кольцо многочленов $\real[x]$ от переменной $x$ с коэффициентами из $\real;$
			\item Кольцо функций $F(M, \real)$ из множества $M$ в $\real$ с поэлементными
			операциями сложения и умножения: \n
			$\forall m \in M{:} \ (f_1 + f_2)(m) = f_1(m) + f_2(m), \quad (f_1 \cdot f_2)(m) = f_1(m) \cdot f_2(m).$
		\end{enumerate}
	\end{exmpls}
	\newpage
	
	\begin{definition}
		Кольцо $R$ называется \textit{ассоциативным} и \textit{коммутативным} соответственно, если 
		\begin{equation*}
			\forall a, b \in R{:} \ ab = ba.
		\end{equation*}
	\end{definition}
	\begin{definition}
		Элемент $a$ кольца $R$ называется \textit{обратимым}, а кольцо \textit{содержащим единицу}, если 
		\begin{equation*}
			\exists b \in R \suchthat ab = ba = 1.
		\end{equation*}
	\end{definition}
	\begin{remark}
		Все обратимые элементы кольца образуют группу по умножению.
	\end{remark}
	\begin{definition}
		Элемент $a$ кольца $R$ называется \textit{левым} (соответственно \textit{правым}) \textit{делителем нуля}, если ${a \neq 0}$ и ${\exists b \neq 0 \in R \suchthat ab = 0}$ (соответственно $ba = 0$).
	\end{definition}
	\begin{remark}
		Если кольцо коммутативно, то множества левых и правых делителей нуля совпадают. Тогда левые и правые делители нуля называются просто «делителями нуля».
	\end{remark}
	\begin{remark}
		Все делители нуля в кольце необратимы.
	\end{remark}
	\begin{proof}
		Пусть $R$~--- кольцо; ${a \neq 0, \ b \neq 0.}$ Если ${ab = 0}$ и ${\exists a^{-1},}$ то ${a^{-1}ab = a^{-1}0 \Longrightarrow b = 0}$~--- противоречие.
	\end{proof}
	\begin{definition}
		Элемент $a$ кольца $R$ называется \textit{нильпотентным (нильпотентом)}, если ${a \neq 0}$ и ${\exists n \in \naturals \suchthat a^n = 0.}$
	\end{definition}
	\begin{remark}
		Всякий нильпотент является делителем нуля.
	\end{remark}
	\begin{definition}
		Кольцо $R$ называется \textit{полем}, если оно коммутативно, ассоциативно, содержит ${1 \neq 0,}$ и любой ненулевой элемент обратим.
	\end{definition}
	\begin{remark}
		В поле не существует делителей нуля.
	\end{remark}
	\begin{exmpls}
		$\rationals, \real, \complex, \algebraic, \integers_2.$
	\end{exmpls}
	\newpage
	\begin{theorem*}
		Кольцо вычетов  $\integers_p$ является полем тогда и только тогда, когда $p$~--- простое число.
	\end{theorem*}
	\begin{proof}
		\
		\newline
		{\textit{Необходимость.}} Если ${n = 1,}$ то ${\integers_n = \{0\}}$ ~--- не поле. \n
		Если ${n > 1,}$ и ${n = m \cdot k,}$  где ${1 < m, k < n,}$ то ${\overline{m} \cdot \overline{k} = \overline{0}} \Longrightarrow$ в $\integers_n$ есть делитель нуля ${\Longrightarrow \integers_n}$~--- не поле. \newline
		{\textit{Достаточность.}} Пусть $p$~--- простое, ${\overline{a} \in \integers_p \, \backslash \, \{\overline{0}\}.}$ \n
		Тогда ${{\NOD(a,p)} = 1} \Longrightarrow {\exists k,l \in \integers \suchthat ak + pl = 1.}$ Значит, ${\overline{a} \cdot \overline{k} + \overline{p} \cdot \overline{l} = \overline{1} \Longrightarrow a \cdot k \equiv 1 \ (\mod p) \Longrightarrow a}$ обратим.
	\end{proof}
    \begin{comment}
	\newpage
	\section{Идеалы колец. Факторкольцо кольца по идеалу. Гомоморфизмы и изоморфизмы колец. Ядро и образ гомоморфизма колец. Теорема о гомоморфизме для колец}
	\setcounter{definition}{0}
	\begin{definition}
		Подмножество $I$ кольца $R$ называется \textit{(двусторонним) идеалом,} если
		\begin{enumerate}
			\setlength\itemsep{0.1em}
			\item $I$~--- подгруппа по сложению;
			\item $\forall a \in I \ \forall r \in R{:} \ ar \in I, ra \in I.$
		\end{enumerate} \n
		\textit{Несобственными} или \textit{тривиальными} идеалами являются $\{0\}$ и $R$. Остальные называются \textit{собственными}.
	\end{definition}
	\begin{definition}
		Множество ${(a) = \{ra \ | \ r \in R\}}$ называется \textit{главным идеалом}, порождаемым элементом $a.$
	\end{definition}
	\begin{exmpl*}
		${(k) = k\integers}$~--- главный идеал в $\integers.$
	\end{exmpl*}
	\begin{remark}
		\ \newline
		$(a) = R \Longleftrightarrow a$ обратим; \newline
		$(a) = 0 \Longleftrightarrow a = 0.$
	\end{remark}
	\begin{definition}
		Если $S$~--- подмножество кольца $R$, то
		\begin{equation*}
			(S) = \{r_1 s_1 + \mathellipsis + r_k s_k \ | \ r_i \in R, s_i \in S\}
		\end{equation*}
		называется \textit{идеалом, порожденным подмножеством $S$.}
	\end{definition}
	Пусть $R$~--- кольцо, $I$~--- его идеал. \n
	Рассмотрим факторгруппу $(R/I, +)$ и введём на ней операцию умножения, полагая, что ${(a + I) \cdot (b + I) = ab + I.}$ \n
	Проверим корректность такого определения:
	\begin{equation*}
		\begin{gathered}
			{a + I = a' + I, \ b + I = b' + I \Longrightarrow a' = a + x, \ b' = b + y, \text{ где } x, y \in I;} \\
			{(a' + I)(b' + I) =  a'b' + I =(a + x)(b + y) + I = ab + \underbrace{ay + xb + xy}_{\in I} + I = ab + I.}
		\end{gathered}
	\end{equation*}
	\begin{remark}
		$R/I$~--- кольцо.
	\end{remark}
	\begin{definition}
		$R/I$ называется \textit{факторкольцом} кольца $R$ по идеалу $I.$
	\end{definition}
	\begin{exmpl*}
		$\integers/n\integers = \integers_n.$
	\end{exmpl*}
	\begin{definition}
		Если $R, S$~--- два кольца, то отображение ${\phi{:} \ R \rightarrow S}$ называется \textit{гомоморфизмом колец,} если
		\begin{equation*}
			\forall a,b \in R{:} \ \phi(a + b) = \phi(a) + \phi(b), \ \phi(ab) = \phi(a) \cdot \phi(b).
		\end{equation*} \n
		Биективный гомоморфизм называется \textit{изоморфизмом}.\n
		${\ker \phi = \{r \in R \ | \ \phi(r) = 0\} \subseteq R;}$ \n
		${\im \phi = \phi(R) \subseteq S.}$
	\end{definition}
	\begin{remark}
		\
		\begin{enumerate}
			\setlength\itemsep{0.25em}
			\item $\ker \phi$ является идеалом $R;$
			\item $\im \phi$~--- подкольцо в $S.$
		\end{enumerate}
	\end{remark}
	\begin{proof}
		\
		\begin{enumerate}
			\setlength\itemsep{0.25em}
			\item $\ker \phi$ является подгруппой в $R$ по сложению, т.к. $\phi$~--- гомоморфизм \n
			абелевых групп. Покажем, что \n ${\forall a \in \ker \phi \ \forall r \in R{:} \ ra \in \ker \phi, ar \in \ker \phi.}$ 
			\begin{equation*}
				\phi(ra) = \phi(r)\phi(a) = \phi(r)0 = 0 \Longrightarrow ra \in \ker \phi, \text{ аналогично для } ar \in \ker \phi.
			\end{equation*}
		\end{enumerate}
	\end{proof}
	\newpage
	\begin{theorem*}[\textbf{О гомоморфизме колец}]
		Пусть ${\phi{:} \ R \rightarrow S}$~--- гомоморфизм колец, тогда
		\begin{equation*}
			R/\ker \phi \cong \im \phi.
		\end{equation*}
	\end{theorem*}
	\begin{proof}
		Пусть ${I = \ker \phi.}$ Тогда из доказательства теоремы о гомоморфизме для групп отображение ${\psi{:} \ R/I \rightarrow \im \phi, \ \psi(a + I) = \phi(a)}$ является изоморфизмом групп по сложению.\n
		Остаётся проверить, что $\psi$~--- гомоморфизм колец:
		\begin{equation*}
			\psi((a + I)(b + I)) = \psi(ab + I) = \phi(ab) = \phi(a)\phi(b) = \psi(a + I)\psi(b + I). \qedhere
		\end{equation*}
	\end{proof}
	\begin{exmpl*}
		Пусть $K$~--- поле, ${a \in K, \quad \phi{:} \ K[x] \rightarrow K, \quad f \mapsto f(a).}$ \n
		Это гомоморфизм, он сюръективен ${(b = \phi(b)).}$ \n
		$\ker \phi = (x - a) \Longrightarrow K[x]/(x - a) \cong K.$
	\end{exmpl*}
    \end{comment}
	\newpage
	\begin{thebibliography}{99}
		\bibitem{sets_theory} Верещагин Н.К., Шень А.
		\textit{Лекции по математической логике и теории алгоритмов. Часть 1. Начала теории множеств}: МЦНМО, 2024.
		\bibitem{kostrikin} Кострикин А.И. \textit{Введение в алгебру}: МЦНМО, 2020.
		\bibitem{kurosh} Курош А.Г. \textit{Курс высшей алгебры}: Лань, 2007.
		\bibitem{winberg} Винберг Э.Б. \textit{Курс алгебры}: МЦНМО, 2019.
		\bibitem{alekseev} Алексеев В.Б. \textit{Теорема Абеля в задачах и решениях}: МЦНМО, 2024.
		\bibitem{atiyah} М. Атья, И. Макдональд. \textit{Введение в коммутативную алгебру}: МЦНМО, 2021.
		\bibitem{lindon} Р. Линдон, П. Шупп. \textit{Комбинаторная теория групп}: Мир, 1980.
		\bibitem{category} С. Маклейн. \textit{Категории для работающего математика}: Физматлит, 2004.
		\bibitem{finite_fields_vid} \href{https://www.youtube.com/watch?v=9PTiaQODKiA}{Иван Аржанцев. \textit{Конечные поля}}
		\bibitem{finite_fields_list} \href{https://youtube.com/playlist?list=PLH3NNipqeM1vcHnP4czWemq5DyBw9MXxn&si=UFBR6POWzySgFqPt}{Алексей Савватеев. \textit{Конечные поля}}
		\bibitem{algebra} \href{https://youtube.com/playlist?list=PLEwK9wdS5g0pxQ2ThGS3ObFwm4h1Ll0Kr&si=AM33kfEJxKFqNR-k}{Роман Авдеев. \textit{Алгебра}}
		\bibitem{geometry_and_groups} \href{https://youtube.com/playlist?list=PLlx2izuC9gjgZsBALx2ZF7IMGoNl07A9l&si=9X5Oosl5a53UTlTb}{Алексей Савватеев. \textit{Геометрия и группы}}
		\bibitem{galois_theory} \href{https://youtube.com/playlist?list=PLgEpoT7yAl9VWebUfw4uOlut1ZaoFjFcb&si=yM6DjTk5Cenofg5U}{Алексей Савватеев. \textit{Теория Галуа}}
		\bibitem{category_theory} \href{https://youtube.com/playlist?list=PLvPsfYrGz3wsX3Z5KuuEVYhijDxhKCDDD&si=eh_MI9GO0mYzxZMU}{Виталий Брагилевский и др. \textit{Теория категорий}}
		\bibitem{for_physicists} \href{https://youtube.com/playlist?list=PLnbH8YQPwKblIpRi0ARO2VadnMwntvF51&si=qXQx9-xB7_arfxB8}{\textit{Элементарное введение в теорию групп (для физиков)}}
	\end{thebibliography}
\end{document}