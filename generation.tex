\newpage
        \section{Порождающие элементы}
        \setcounter{definition}{0}
        \begin{definition}
            Подгруппа в группе $G$ называется порождённой подмножеством $S$, если эта группа
            есть множество элементов вида ${g_{i_1}^{\epsilon_1},\ldots,g_{i_k}^{\epsilon_k}}$, где ${g_{i_p} \in S, \epsilon_p \in \{\pm1\}.}$ Легко заметить, что это наименьшая подгруппа в $G$, содержащая $S$.
        \end{definition}
        \begin{statement}
            Группа $\alternating_n$ порождается:
            \
            \begin{enumerate}
            \setlength\itemsep{0.1em}
                \item парами транспозиций; 
                \item тройными циклами;
                \item парами независимых транспозиций при $n \geqslant 5$.
            \end{enumerate}
        \end{statement}
        \begin{proof}
            \
            \begin{enumerate}
            \setlength\itemsep{0.1em}
                \item ${\forall \sigma \in \symmetrical_n\suchthat \sigma = \tau_1 \ldots \tau_k,}$ где $\tau_i$~--- транспозиция. Если $\sigma$ чётная, то ${k = 2s.}$ Из этого следует, что ${\sigma = (\tau_1 \tau_2)\ldots(\tau_{2s-1}\tau_{2s}).}$
                \item Выразим пары транспозиций через тройные циклы:
                \begin{equation*}
                    (ij)(ij) = e, \ (ij)(jk) = (ijk), \ (ij)(kl) = (ijk)(jkl).
                \end{equation*}
                \item Выразим пару зависимых транспозиций через пары независимых транспозиций:
                \begin{equation*}
                    (ij)(jk) = ((ij)(lm))((jk)(lm)), \ \ l,m \in \{i, j, k\}. \qedhere
                \end{equation*}
            \end{enumerate}
        \end{proof}
        \begin{definition}
            \textit{Задание (копредставление или генетический код) группы}~--- один из методов описания группы.
             Пусть подмножество $S$ группы $G$ порождает её. При такой кодировке конкатенация слов соответствует умножению элементов группы, а значит, теоретически вся групповая структура задаётся информацией о том, какие пары таких слов представляют один и тот же элемент группы $G$. Такие пары называются \textit{соотношениями}.  Метод состоит в том, чтобы указать (по возможности небольшой) список $R$ определяющих соотношений, которого, с учётом заранее оговоренных правил вывода, хватит для хранения полной информации о группе. В этом случае пишут ${G \cong \langle S \ | \ R \rangle.}$
        \end{definition}