\newpage
        \section{Классы сопряжённости}
        \setcounter{definition}{0}
        \begin{definition}
            Пусть $G$~--- группа. Элементы ${a,b \in G}$ называются \textit{сопряжёнными} и обозначаются ${a \sim b}$, если ${\exists g \in G\suchthat a = gbg^{-1}.}$
        \end{definition}
        \begin{definition}
            \textit{Классом сопряжённости} элемента $a$ группы $G$ называется множество
            \begin{equation*}
                C_G(a) \deq \{b \in G \ | \ a \sim b\}.
            \end{equation*}
        \end{definition}
        \begin{lemma}
            Пусть $G$~--- группа.
            \begin{enumerate}
            \setlength\itemsep{0.1em}
                \item Отношение сопряжённости является отношением эквивалентности.
                \item ${C_G(a) = \{a\} \Leftrightarrow a \in Z(G).}$
            \end{enumerate}
        \end{lemma}
        \begin{proof}
            \
            \begin{enumerate}
            
                \item Проверим свойства отношения эквивалентности:
                \begin{itemize}
                \setlength\itemsep{0.1em}
                    \item Рефлексивность: ${a = eae^{-1} \Rightarrow a \sim a;}$
                    \item Симметричность: ${a = gba^{-1} \Leftrightarrow b = g^{-1}ag;}$
                    \item Транзитивность: ${(a = gbg^{-1}, \, b = hch^{-1} \Rightarrow a = ghch^{-1}g^{-1}) \Rightarrow (a \sim b, \, b \sim c \Rightarrow a \sim c).}$
                \end{itemize}
                \item ${C_G(a) = \{a\} \Leftrightarrow \forall g \in G\suchthat gag^{-1} = a \Leftrightarrow a \in Z(G).}$ \qedhere
            \end{enumerate}
        \end{proof}
        \begin{lemma}
            Пусть $G$~--- группа, ${a,b \in G.}$ \\
            Тогда ${b \in C_G(a) \Rightarrow \ord(b) = \ord(a).}$
        \end{lemma}
        \begin{proof}
            Пусть ${b = gag^{-1}, \ a^n = e.}$ \\
            Тогда ${b^n = (gag^{-1})^n = ga^ng^{-1} = gg^{-1} = e}$ и наоборот из симметричности сопряжённости $\Rightarrow$ минимальные показатели совпадают.
        \end{proof}
        \begin{definition}
            \textit{Централизатором} элемента $a$ группы $G$ называется множество
            \begin{equation*}
                Z_G(a) \deq \{g \in G \ | \ ga = ag\}.
            \end{equation*}
        \end{definition}
        \begin{remark}
            ${Z_G(a) < G.}$
        \end{remark}
        \newpage
        \begin{statement}
            Пусть $G$~--- конечная группа, ${a \in G.}$ Тогда ${|C_G(a)| = |\frac{|G|}{|Z_G(a)|}|.}$ В частности, ${|G| \divby |C_G(a)|.}$
        \end{statement}
        \begin{proof}
            Пусть $\sfrac{G}{Z_G(a)}$~--- множество левых смежных классов (не факторгруппа, т.к. ${Z_G(a)}$ не обязательно нормальна). Достаточно установить биекцию ${\sfrac{G}{Z_G(a)} \leftrightarrow C_G(a).}$\newline
            Определим отображение ${\sfrac{G}{Z_G(a)} \rightarrow C_G(a),}$ ${gZ_G(a) \mapsto gag^{-1}.}$ Проверим:
            \begin{enumerate}
            \setlength\itemsep{0.1em}
                \item Корректность:
                \begin{equation*}
                    \begin{gathered}
                        h \in Z_G(a) \Rightarrow ghZ_G(a) \mapsto (gh)a(gh)^{-1} = ghah^{-1}g^{-1} = gahh^{-1}g^{-1} = \\
                        = gag^{-1} \mapsfrom gZ_G(a);
                    \end{gathered}
                \end{equation*}
                \item Сюръективность: по построению;
                \item Инъективность:\
                \begin{equation*}
                    gag^{-1} = g'a(g')^{-1} \Leftrightarrow ag^{-1}g' = g^{-1}g'a \Leftrightarrow g^{-1}g' \in Z_G(a) \Leftrightarrow g' \in Z_G(a). \qedhere
                \end{equation*}
            \end{enumerate}
        \end{proof}