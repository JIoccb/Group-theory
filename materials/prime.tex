\newpage
        \section{Простые группы}
        \setcounter{definition}{0}
        \begin{definition}
            Группа называется \textit{простой}, если в ней нет нетривиальных нормальных подгрупп.
        \end{definition}
        \begin{theorem}
            Пусть $G$~--- конечная группа, тогда существует ряд подгрупп ${G > H_1 > H_2 > \ldots > H_k = \{e\},}$ такой что ${H_{i + 1} \lhd H_i, \ \sfrac{H_i}{H_{i+1}}}$ проста ${\forall i = \overline{1,k}.}$
        \end{theorem}
        \begin{definition}
            Пусть $G$~--- группа, ${N \lhd G,}$ ${F = G/N.}$ Тогда говорят, что $G$~--- \textit{расширение} группы $N$ с помощью подгруппы $F$.
        \end{definition}
        \begin{remark}
            Имеет место цепочка гомоморфизмов:
            \begin{equation*}
                N \xhookrightarrow{i} G \xlongrightarrow{\pi} G/N = F.
            \end{equation*}
        \end{remark}
        \begin{consequence*}
            Любая конечная группа получается цепочкой расширений при помощи простых групп.
        \end{consequence*}
        \begin{exmpls}
            Не всегда тем, что $G$~--- расширение $N$ с помощью $F$, $G$ определяется однозначно. Например, пусть ${N \cong \integers_3}$, ${F \cong \integers_2}$. Тогда: 
            \begin{enumerate}
            \setlength\itemsep{0.1em}
                \item ${N = \integers_3, \ F = \integers_2 \Rightarrow G = \integers_3 \oplus \integers_2}$~--- абелева.
                \item ${N = \alternating_3 \cong \integers_3, \ F = \symmetrical_3/\alternating_3 \cong \integers_2 \Rightarrow G = \symmetrical_3}$~--- неабелева.
            \end{enumerate}
        \end{exmpls}
        \begin{remark}
            Абелева группа $A$ проста ${\Leftrightarrow A \cong \integers_p,}$ где $p$~--- простое.
        \end{remark}
         \begin{theorem}
             Группа $\alternating_n$ проста при ${n \geqslant 5.}$
         \end{theorem}
         \begin{theorem*}[\textbf{о классификации конечных простых групп}]
             Любая конечная простая группа изоморфна либо одной из 26 спорадических групп, либо принадлежит одному из следующих трёх семейств:
             \begin{itemize}
             \setlength\itemsep{0.1em}
                 \item ${\integers_p,}$ $p$~--- простое;
                 \item ${\alternating_n,}$ ${n \geqslant 5;}$
                 \item простые группы типа Ли.
             \end{itemize}
         \end{theorem*}