\newpage
	\section{Группы движений}
	\setcounter{definition}{0}
	\begin{definition}
		Упорядоченная пара ${(M, d),}$ состоящая из множества $M$ и отображения ${d : M \times M \rightarrow \real,}$  называется \textit{метрическим пространством}, если ${\forall x, y \in M{:}}$
		\begin{enumerate}
			\setlength\itemsep{0.1em}
			\item ${d(x, y) = 0 \Leftrightarrow x = y}$ (\textit{аксиома тождества});
			\item ${d(x, y) \geqslant 0}$ (\textit{аксиома неотрицательности});
			\item ${d(x, y) = d(y, x)}$ (\textit{аксиома симметричности});
			\item ${d(x, y) + d(y, z) \geqslant d(x, y)}$ (\textit{аксиома} или \textit{неравенство треугольника}).
		\end{enumerate}
	\end{definition}
	\begin{definition}
		Пусть $X$ и $Y$~--- метрические пространства. Отображение ${f : X \rightarrow Y}$ называется \textit{изометрией}, если оно сохраняет расстояние между точками:
		\begin{equation*}
			\forall x, x' \in X\suchthat |f(x) - f(x')|_Y = |x - x'|_X,
		\end{equation*} \n
		Если ${X = Y,}$ $f$ называют \textit{движением}.
	\end{definition}
	\begin{definition}
		Движение называют \textit{собственным}, если оно сохраняет \textit{ориентацию} пространства.
	\end{definition}
	\begin{definition}
		Пусть $E$~--- \textit{евклидово аффинное пространство} и ${F \subseteq E}$~--- геометрическая фигура. \textit{Группой движений (изометрий)} ${\Isom(F)}$ фигуры $F$ называется множество тех движений аффинного пространства $E$, которые  переводят фигуру $F$ в себя:
		\begin{equation*}
			\Isom(F) \deq \{\phi: E \rightarrow E \ | \ \phi\textrm{~--- движение}, \ \phi(F) = F\}.
		\end{equation*} 
		В качестве групповой операции рассматривается операция композиции движений.
	\end{definition}
	\begin{remark}
		Группа собственных движений ${\Isom(F)^+}$ является подгруппой группы движений ${\Isom(F)}$ фигуры $F.$
	\end{remark}
	\newpage
	\begin{definition}
		Группа движений правильного $n$-угольника ${\Delta_n \subset \real^2}$ называется \textit{диэдральной группой} $\dihedral_n{:}$
		\begin{equation*}
			\dihedral_n \deq \Isom(\Delta_n).
		\end{equation*}
	\end{definition}
	\begin{statement}
		${|\dihedral_n| = 2n.}$
	\end{statement}
        \begin{proof}
            Есть всего 2 вида движений:
		\begin{enumerate}
			\setlength\itemsep{0.1em}
			\item $n$ вращений относительно центра на угол, кратный $\frac{2\pi}{n}$ (вращение на угол $\phi$ обозначается $\Rot_\phi$);
			\item $n$ симметрий относительно осей симметрии (симметрия относительно прямой $l$ обозначается $\Sym_l$). \n В случае нечётного $n$ любая ось симметрии проходит через центр $\Delta_n$ и одну из вершин, в случае чётного $n$ любая ось симметрии проходит либо через противоположные вершины, либо через середины противоположных сторон. \qedhere
		\end{enumerate}
        \end{proof}
	\begin{remark}
		Группа собственных движений $\Delta_n$ содержит только повороты:
		\begin{equation*}
			\Isom(\dihedral_n)^+ = \{\Rot_{\frac{2\pi k}{n}}\}, \ k = \overline{0, \ n - 1}.
		\end{equation*}
	\end{remark}
	\begin{exmpl*}
		Таблица Кэли группы $\dihedral_4$ квадрата $ABCD{:}$
		\begin{center}
			\begin{tabular}{c|c|c|c|c|c|c|c|c}
				$\circ$ & $\id$ & $\Rot_{\frac{\pi}{2}}$ & $\Rot_{\pi}$ & $\Rot_{\frac{3\pi}{2}}$ & 
				$\Sym_h$ &
				$\Sym_v$ &
				$\Sym_{AC}$ &
				$\Sym_{BD}$ \\
				\hline
				$\id$ & $\id$ & $\Rot_{\frac{\pi}{2}}$ & $\Rot_{\pi}$ & $\Rot_{\frac{3\pi}{2}}$ & 
				$\Sym_h$ &
				$\Sym_v$ &
				$\Sym_{AC}$ &
				$\Sym_{BD}$ \\
				\hline
				$\Rot_{\frac{\pi}{2}}$ & $\Rot_{\frac{\pi}{2}}$ & $\Rot_{\pi}$ & $\Rot_\frac{3\pi}{2}$ & $\id$ & $\Sym_{BD}$ & $\Sym_{AC}$ & $\Sym_h$ & $\Sym_v$ \\
				\hline
				$\Rot_{\pi}$ & $\Rot_{\pi}$ & $\Rot_\frac{3\pi}{2}$ & $\id$ & $\Rot_\frac{\pi}{2}$ & $\Sym_v$ & $\Sym_h$ & $\Sym_{BD}$ & $\Sym_{AC}$ \\ \hline
				$\Rot_\frac{3\pi}{2}$ & $\Rot_\frac{3\pi}{2}$ & $\id$ & $\Rot_\frac{\pi}{2}$ & $\Rot_{\pi}$ & $\Sym_{AC}$ & $\Sym_{BD}$ & $\Sym_v$ & $\Sym_h$ \\
				\hline
				$\Sym_h$ & $\Sym_h$ & $\Sym_{AC}$ & $\Sym_v$ & $\Sym_{BD}$ & $\id$ & $\Rot_{\pi}$ & $\Rot_\frac{\pi}{2}$ & $\Rot_\frac{3\pi}{2}$
				\\
				\hline
				$\Sym_v$ & $\Sym_v$ & $\Sym_{BD}$ & $\Sym_h$ & $\Sym_{AC}$ & $\Rot_{\pi}$ & $\id$ & $\Rot_\frac{3\pi}{2}$ & $\Rot_\frac{\pi}{2}$
				\\
				\hline
				$\Sym_{AC}$ & $\Sym_{AC}$ & $\Sym_v$ & $\Sym_{BD}$ & $\Sym_h$ & $\Rot_\frac{3\pi}{2}$ & $\Rot_\frac{\pi}{2}$ & $\id$ & $\Rot_{\pi}$
				\\
				\hline
				$\Sym_{BD}$ & $\Sym_{BD}$ & $\Sym_h$ & $\Sym_{AC}$ & $\Sym_v$ & $\Rot_\frac{\pi}{2}$ & $\Rot_\frac{3\pi}{2}$ & $\Rot_{\pi}$ & $\id$\\
			\end{tabular}
		\end{center} \n
	\end{exmpl*}